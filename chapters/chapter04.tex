\chapter{Introductie Python}
\begin{fquote}[John Cleese][Monty Python][1975]
	And now for something completely different.
\end{fquote}

We stappen nu een compleet nieuwe programmeer-wereld in, namelijk die van \textit{Python}. \textit{Python} is een programmeertaal van hogere orde ten opzichte van \textit{C}. Hoe hoger de orde van een programmeertaal, des te verder de taal weg staat van daadwerkelijke machine-instructies. \newline
Python is een favoriete taal voor menig programmeur, omdat het redelijk makkelijk te leren is, de code compact en leesbaar is, voor een enorm scala aan projecten in te zetten is (van websites en kunstmatige intiligentie tot scripts en Raspberry Pi's), en de enorme gemeenschap die de taal gebruikt en ook verder uitbreidt d.m.v. softwarebibliotheken (hierover later meer). \newline

Zoals altijd komen al deze voordelen ook met enige nadelen (waar omheen gewerkt kan worden). \textit{Python} is bijv. een stuk langzamer dan \textit{C} (afhankelijk van de toepassing zo'n 2 tot 100 keer), vandaar dat we met dit vak afstappen van de inmiddels bekende \textit{Arduino} en overstappen naar de \textit{Raspberry Pi} en je eigen laptop/PC.

\section{Installatie}\index{Installatie}
\begin{exercise}
Installeer de laatste versie van Python op je eigen laptop via: \newline
\url{https://www.python.org/downloads/} \newline
Let er bij het installeren op dat je alle vinkjes aanzet (vooral $pip$ is een belangrijke).
\end{exercise}

\section{Interpreter}\index{Interpreter}

\section{Draaien van scripts}\index{Draaien van scripts}

\section{Editor}\index{Editor}

\section{Data Types}\index{Data Types}

\section{If/Else statements}\index{If/Else statements}

\section{Input/Output}\index{Input/Output}

\section{Opdrachten}\index{Opdrachten}
\begin{exercise}
$\\$
\end{exercise}

\begin{exercise}
$\\$
\end{exercise}

\begin{exercise}
$\\$
\end{exercise}

\begin{exercise}
$\\$
\end{exercise}

\begin{exercise}
$\\$
\end{exercise}

\begin{exercise}
$\\$
\end{exercise}

\begin{exercise}
$\\$
\end{exercise}

\begin{exercise}
$\\$
\end{exercise}

