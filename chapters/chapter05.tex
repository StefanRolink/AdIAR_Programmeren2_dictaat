\chapter{Python op de Pi}

\section{Introductie linux terminal}\index{Introductie linux terminal}

\section{Editor}\index{Editor}

\section{Pi header}\index{Pi header}

\section{GPIO}\index{GPIO}

\section{Loops}\index{Loops}
Als we bepaalde onderdelen van onze code vaker uit willen voeren, kunnen we dat doen aan de hand van \textit{loops}. In \textit{C} hadden we al kennisgemaakt met de \pyth{for}-loop. Deze zit gelukkig ook in \textit{Python}, maar werkt wel een beetje anders. We gebruiken deze vaak in combinatie met de funtie \pyth{range()}:
\begin{python}
for x in range(0, 10, 1):
	print(x)
\end{python}
In het bovenstaande voorbeeld roepen we de functie \pyth{range()} aan met 3 argumenten. De eerste geeft het startgetal voor $x$ aan, de tweede bij welke waarde van $x$ de loop moet stoppen, en de derde en laatste met welke hoeveel we $x$ we moeten verhogen bij elke nieuwe iteratie. Dit voorbeeld print dus de getallen $0$ t/m $9$ op het scherm. 

\begin{remark}
De functie \pyth{range()} kun je op $3$ verschillende manieren aanroepen:
\begin{enumerate}
\item[-] \pyth{range(stop)}: Met enkel $1$ argument, de stop waarde. $start=0, step=1$.
\item[-] \pyth{range(start, stop)}: Met $2$ argumenten, voor start en stop. $step=1$.
\item[-] \pyth{range(start, stop, stap)}: En $3$, zoals in het voorbeeld.
\end{enumerate}
In het bovenstaande stukje voorbeeldcode kan dus \pyth{range(0, 10, 1)} worden vervangen door \pyth{range(10)}, want die levert dezelfde functionaliteit.
\end{remark}

Als tweede voorbeeld een loop die van $2$ t/m $25$ loopt, met stapjes van $3$:
\begin{python}
for x in range(2, 25, 3):
	print(x)
\end{python}
Dit print het volgende uit: $2, 5, 8, 11, 14, 17, 20, 23$. De \pyth{for}-loop stopt na $23$, omdat $23+3 = 26$, wat hoger is dan $25$, de stop waarde.


\section{Huiswerkopdrachten}\index{Huiswerkopdrachten}
\begin{exercise}
$\\$
\end{exercise}

\begin{exercise}
$\\$
\end{exercise}

\begin{exercise}
$\\$
\end{exercise}

\begin{exercise}
$\\$
\end{exercise}

\begin{exercise}
$\\$
\end{exercise}

\begin{exercise}
$\\$
\end{exercise}

\begin{exercise}
$\\$
\end{exercise}

\begin{exercise}
$\\$
\end{exercise}

