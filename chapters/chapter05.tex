\chapter{Python op de Pi}

\section{Introductie linux terminal}\index{Introductie linux terminal}

\section{Editor}\index{Editor}

\section{Pi header}\index{Pi header}

\section{GPIO}\index{GPIO}

\section{Loops}\index{Loops}
Als we bepaalde onderdelen van onze code vaker uit willen voeren, kunnen we dat doen aan de hand van \textit{loops}. In \textit{C} hadden we al kennisgemaakt met de \pyth{for}-loop. Deze zit gelukkig ook in \textit{Python}, maar werkt wel een beetje anders. We gebruiken deze vaak in combinatie met de funtie \pyth{range()}:
\begin{python}
for x in range(0, 10, 1):
	print(x)
\end{python}
In het bovenstaande voorbeeld roepen we de functie \pyth{range()} aan met 3 argumenten. De eerste geeft het startgetal voor $x$ aan, de tweede bij welke waarde van $x$ de loop moet stoppen, en de derde en laatste met welke hoeveel we $x$ we moeten verhogen bij elke nieuwe iteratie. Dit voorbeeld print dus de getallen $0$ t/m $9$ op het scherm:
\begin{python}
0
1
2
3
4
5
6
7
8
9
\end{python}

\begin{remark}
De functie \pyth{range()} kun je op $3$ verschillende manieren aanroepen:
\begin{enumerate}
\item[-] \pyth{range(stop)}: Met enkel $1$ argument, de stop waarde. $start=0, step=1$.
\item[-] \pyth{range(start, stop)}: Met $2$ argumenten, voor start en stop. $step=1$.
\item[-] \pyth{range(start, stop, stap)}: En $3$, zoals in het voorbeeld.
\end{enumerate}
In het bovenstaande stukje voorbeeldcode kan dus \pyth{range(0, 10, 1)} worden vervangen door \pyth{range(10)}, want die levert dezelfde functionaliteit.
\end{remark}

Als tweede voorbeeld een loop die van $2$ t/m $25$ loopt, met stapjes van $3$:
\begin{python}
for x in range(2, 25, 3):
	print(x)
\end{python}
Dit print het volgende uit: $2, 5, 8, 11, 14, 17, 20, 23$. De \pyth{for}-loop stopt na $23$, omdat $23+3 = 26$, wat hoger is dan $25$, de stop waarde. \newline\newline

Naast de \pyth{for}-loop, bestaat er ook de \pyth{while}-loop. Deze zit ook in \textit{C}, maar hebben we destijds niet gebruikt. Wellicht is het handig om te weten dat hij bestaat, en te snappen hoe je 'm gebruikt:
\begin{python}
i = 1
while i < 6:
	print(i)
	i += 1
\end{python}
In het bovenstaand voorbeeld wordt de code in de \pyth{while}-loop uitgevoerd \textit{zolang} de conditie \pyth{i < 6} gelijk is aan \pyth{True}. We beginnen met \pyth{i = 1}, en tellen er elke keer \pyth{1} bij op, waardoor de uitvoer van het programma $1$ t/m $5$ is. \par

Met een \pyth{while}-loop is ook vrij makkelijk een oneindige loop te maken, vergelijkbaar met de \textit{loop()}-functie bij de \textit{Arduino}:
\begin{python}
while True:
	print('en door..')
\end{python}

\begin{remark}
Het is bij zo'n loop wel handig om te weten hoe je 'm weer stopt. Want in de meeste gevallen maak je 'm per ongeluk. Bij een \textit{IDE} heb je naast een run knopt (groen driehoekje), vaak ook een stop knop (rood vierkantje). In de terminal kun je de toetscombinatie \text{Ctrl+C} gebruiken. 
\end{remark}

\newpage

\section{Huiswerkopdrachten}\index{Huiswerkopdrachten}
\begin{exercise}
Schrijf een programma waarin je met een \pyth{while}-loop de eerste $10$ getallen van de tafel van $5$ print.
\end{exercise}

\begin{exercise}
Schrijf een programma waarin je de gebruiker vraagt om een getal, en geef daarna de eerste $10$ getallen van de tafel voor dat getal. Gebruik hiervoor een \pyth{for}-loop.
\end{exercise}

\begin{exercise}
Schrijf een programma die de getallen van $-10$ t/m $-1$ op scherm print. Gebruik hiervoor alleen de \pyth{range()}-functie in de \pyth{for}-loop (en natuurlijk een \pyth{print()} ;) ).
\end{exercise}

\begin{exercise}\label{exc5:exc4}
Vraag aan de gebruiker een getal $l$, en print daarna een rij van sterretjes. Dus bij $l=7$, geeft het de volgende output:
\begin{python}
*******
\end{python}
\textbf{TIP:} Als je niet wilt dat een \pyth{print}-functie elke keer een nieuwe regel begint, kun je dit veranderen door het \textit{end}-argument te gebruiken: \pyth{print('*', end='')}. 
\end{exercise}


\begin{exercise}
Breid Vraag \ref{exc4:exc1} uit. Vraag aan de gebruiker nu $2$ getallen: $h$ en $l$. Print daarna een vierkant van sterretjes ter grootte van $h \times l$. Dus bij $h=3$ en $l=5$, geeft het de volgende output:
\begin{python}
*****
*****
*****
\end{python}
\textbf{TIP:} Je kunt loops in loops zetten. 
\end{exercise}

\begin{exercise}
$\\$
\end{exercise}

\begin{exercise}
$\\$
\end{exercise}

\begin{exercise}
$\\$
\end{exercise}

\begin{exercise}
$\\$
\end{exercise}

